
\subsection{Vector Carry-less Multiply}

The following instructions support vectorised Carry-less multiply
operations.
These are essential to support the Galois/Counter Mode (GCM) \cite{nist:gcm}
block cipher mode of operation.
GCM is the only mandatory mode of operation in the TLS 1.3
ciphersuite \cite[Section 9.1]{tls:1.3}.


\subsubsection{Single-Width Carry-less Multiply}
\label{sec:vector:clmul:hilo}

\begin{cryptoisa}
vclmul.vv   vrd, vrs1, vrs2 // vrd[i]  = vrs1[i] * vrs2[i] (low SEW-bits)
vclmul.vs   vrd, vrs1, vrs2 // vrd[i]  = vrs1[0] * vrs2[i] (low SEW-bits)
vclmulh.vv  vrd, vrs1, vrs2 // vrd[i]  = vrs1[i] * vrs2[i] (hi  SEW-bits)
vclmulh.vs  vrd, vrs1, vrs2 // vrd[i]  = vrs1[0] * vrs2[i] (hi  SEW-bits)
\end{cryptoisa}

These instructions multiply corresponding \SEW-bit elements
to produce a $2*\SEW$-bit result.
{\tt vclmul.*} writes back the low \SEW bits of the result to the
element in \vrd.
{\tt vclmulh.*} writes back the high \SEW bits of the result to the
element in \vrd.

The vector-vector ({\tt *.vv}) instructions multiply corresponding
elements in each vector.
The vector-scalar ({\tt *.vs}) instructions multiply the zeroth
element of \vrs{1} with the $i'th$ element of \vrs{2}.
The results of each instruction are always written back to the $i'th$
element of \vrd.


\subsubsection{Single-Width Carry-less Multiply with Accumulate}
\label{sec:vector:clmul:accumulating}

\begin{cryptoisa}
vclmacc.vv  vrd, vrs1, vrs2 // vrd[i] += vrs1[i] * vrs2[i] (low SEW-bits)
vclmacc.vs  vrd, vrs1, vrs2 // vrd[i] += vrs1[0] * vrs2[i] (low SEW-bits)
vclmacch.vv vrd, vrs1, vrs2 // vrd[i] += vrs1[i] * vrs2[i] (hi  SEW-bits)
vclmacch.vs vrd, vrs1, vrs2 // vrd[i] += vrs1[0] * vrs2[i] (hi  SEW-bits)
\end{cryptoisa}

These instructions multiply \SEW-bit elements
to produce a $2*\SEW$-bit intermediate.
{\tt vclmul.*} adds the low \SEW bits of the intermediate to the
\SEW-bit element \vrd, and writes the result back to \vrd.
{\tt vclmulh.*} adds the high \SEW bits of the intermediate to the
\SEW-bit element \vrd, and writes the result back to \vrd.
Note that addition in this context is carry-less, and is simply an {\tt xor}
operation.

The vector-vector ({\tt *.vv}) instructions operate on corresponding
elements in each vector.
The vector-scalar ({\tt *.vs}) instructions operate on the zeroth
element of \vrs{1} with the $i'th$ elements of \vrs{2} and \vrd.


\subsubsection{Widening Carry-less Multiply}
\label{sec:vector:clmul:widening}

\begin{cryptoisa}
vwclmul.vv  vrd, vrs1, vrs2 // vrd[i] = vrs1[i] * vrs2[i], LMUL(vrd)=2
vwclmul.vs  vrd, vrs1, vrs2 // vrd[i] = vrs1[0] * vrs2[i], LMUL(vrd)=2
\end{cryptoisa}

These instructions multiply \SEW-bit elements
to produce a $2*\SEW$-bit result.
The $2*\SEW$-bit result is written back in its entirety to \vrd,
where \vrd is treated as a double-width register such that
$\LMUL=2$ and $\EEW(\vrd)=2*\SEW$.

The vector-vector ({\tt *.vv}) instructions operate on corresponding
elements in each vector.
The vector-scalar ({\tt *.vs}) instructions operate on the zeroth
element of \vrs{1} with the $i'th$ elements of \vrs{2} and \vrd.


\subsubsection{Widening Carry-less Multiply With Accumulate}
\label{sec:vector:clmul:widening}

\begin{cryptoisa}
vwclmacc.vv vrd, vrs1, vrs2 // vrd[i] += vrs1[i] * vrs2[i], LMUL(vrd)=2
vwclmacc.vs vrd, vrs1, vrs2 // vrd[i] += vrs1[0] * vrs2[i], LMUL(vrd)=2
\end{cryptoisa}

These instructions multiply \SEW-bit elements
to produce a $2*\SEW$-bit intermediate.
The $2*\SEW$-bit intermediate is added to the $2*\SEW$-bit element in \vrd.
The result is then written back to \vrd.
\vrd is treated as a double-width register such that
$\LMUL=2$ and $\EEW(\vrd)=2*\SEW$.
Note that addition in this context is carry-less, and is simply an {\tt xor}
operation.

The vector-vector ({\tt *.vv}) instructions operate on corresponding
elements in each vector.
The vector-scalar ({\tt *.vs}) instructions operate on the zeroth
element of \vrs{1} with the $i'th$ elements of \vrs{2} and \vrd.

