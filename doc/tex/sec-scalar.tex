
As per the RISC-V Cryptographic Extensions Task Group charter:
``{\em The committee will also make ISA extension proposals for lightweight
scalar instructions for 32 and 64 bit machines that improve the performance
and reduce the code size required for software execution of common algorithms
like AES and SHA and lightweight algorithms like PRESENT and GOST}".

\bigskip

For context, some these instructions have been developed based on academic
work at the University of Bristol as part of the XCrypto project
\cite{MPP:19},
and work by
Paris Telecom on acceleration of lightweight block ciphers
\cite{TGMGD:19}.

\question{
Implementation Diversity.
There are often many different ways of implementing certain
standardised cryptographic algorithms (AES T-Table v.s. packed for example).
Given that different implementation styles can have security implications
as well, what should the TG policy be on enabling implementation diversity?
By encouraging a particular style of implementation, more specific and
light-weight instructions can be defined to accelerate specifically that
implementation (See \cite{TG:06}).
This may come with flexibility costs, and possibly limit implementers
choices in terms of security and attack countermeasures.
}

\question{
Chicken and Egg Problems. While security must always be the first design
criteria for a new cipher, cryptographers can weight their choice of
primitives by how well they are supported by existing micro-processors.
If suddenly a new primitive becomes well supported by new
architectures, how does this impact cipher design?
Is this a motivation to include some level of generally / speculatively
useful stuff?
See \cite{block:salsa20, LSYRR:04}.
Likewise, criteria for ``lightweight" ciphers might include suitability
for hardware.
This might mean narrow data widths (e.g. SPARX \cite{DPUVGB:16})
which though excellent for hardware, are less than a CPU word-width,
making operations like addition, shifts and rotate {\em more}
awkward, not less.
}


% ============================================================================

\import{./}{sec-scalar-bitmanip.tex}
\import{./}{sec-scalar-lut4.tex}
\import{./}{sec-scalar-mparith.tex}
\import{./}{sec-scalar-aes.tex}
\import{./}{sec-scalar-sha2.tex}
\import{./}{sec-scalar-sha3.tex}
\import{./}{sec-scalar-ildst.tex}

% ============================================================================

\subsection{Micro-architectural Recommendations}

\todo{Macro-op fusion suggestions, side-channel considerations.}

% ============================================================================

